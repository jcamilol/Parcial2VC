\begin{Ejc}
   Calcule usando residuos:
   $$
   \int_{-\infty}^{\infty}\frac{dx}{ax^2+bx+c} 
   $$
   y
   $$
   \int_{-\infty}^{\infty}\frac{dx}{\left( ax^2+bx+c\right)^2}
   $$
   donde $a, b$ y $c$ son reales tales que $b^2<4ac$.
\end{Ejc}
\textit{Solución.}{
   \begin{itemize}
      \item[\textbf{I.}]$\int_{-\infty}^{\infty}\frac{dx}{ax^2+bx+c}$\\
         Llamemos $f(z)=1/(az^2+bz+c)$. Como $4ac>b^2\geq0$, tenemos $a\neq 0$. Así, tomando
         $$
         z_1=-\frac{b}{2a}+i\frac{\sqrt{4ac-b^2}}{2|a|}\phantom{000}\text{y}\phantom{000}z_2=-\frac{b}{2|a|}-i\frac{\sqrt{4ac-b^2}}{2a}=\overline{z_1},
         $$
         tenemos
         $$
         f(z)=\frac{1}{a(z-z_1)(z-z_2)}.
         $$
         Llamando $q(z)=a(z-z_1)(z-z_2)$, como $4ac-b^2\neq 0$, tenemos que $q(z)$ no tiene ceros reales, y $z_1$ es su único cero sobre el eje real. Además, como la función $1/(a(z-z_2))$ es analítica y no nula en $z_1$, se tiene que $z_1$ es un polo simple de $f(z)$, y
         $$
         \begin{aligned}
            B:=&=\mathop{\mathrm{Res}}\limits_{z=z_1}f(z)\\
               &=\frac{1}{a(z_1-z_2)}\\
               &=\frac{1}{a(z_1-\overline{z_1})}\\
               &=\frac{1}{2ia\cdot\text{Im}(z_1)}\\
               &=\frac{1}{2ia\frac{\sqrt{4ac-b^2}}{2|a|}}\\
               &=\frac{|a|}{ia\sqrt{4ac-b^2}}\\
               &=\sgn(a)\frac{1}{i\sqrt{4ac-b^2}}.
         \end{aligned}
         $$
         Tomamos $R>|z_1|$ y llamamos $C_R$ a la curva $Re^{i\theta}$ con $\theta\in[0,\pi]$, y $C$ a la curva que consta del eje real de $-R$ a $R$ junto a $C_R$. Por el teorema de los residuos de Cauchy tenemos
         $$
         \int_{-R}^{R}f(x)dx+\int_{C_R}f(z)dz=\int_{C}f(z)dz=2\pi i B,
         $$
         y por tanto,
         $$
         \lim_{R\to\infty}\int_{-R}^{R}f(x)dx=2\pi i B + \lim_{R\to\infty}\int_{C_R}f(z)dz.
         $$
%%%%%%%%%%%%%%%%%%%%%%%%%%%%%%%%%%%%%%%%%%%%%%%%%%%%%%%%%%%
         Para cada $z\in C_R$ tenemos
         $$
         \begin{aligned}
            |z-z_1|&\geq ||z|-|z_1||\\
                   &=|R-|z_1||\\
                   &=R-|z_1|\\
         \end{aligned}
         $$
         (pues $R>|z_1|$), y
         $$
         \begin{aligned}
            |z-z_2|&\geq||z|-|z_2||\\
                   &=|R-|z_2||\\
                   &=|R-|\overline{z_1}||\\
                   &=|R-|z_1||\\
                   &=R-|z_1|.
         \end{aligned}
         $$
         Por tanto
         $$
         \frac{1}{(R-|z_1|)^2}\geq\frac{1}{|z-z_1||z-z_2|},
         $$
         de modo que
         $$
         \begin{aligned}
            |f(z)|&=\left|\frac{1}{a(z-z_1)(z-z_2)}\right|\\
                  &=\frac{1}{|a||z-z_1||z-z_2|}\\
                  &\leq\frac{1}{|a|(R-|z_1|)^2},
         \end{aligned}
         $$
         con lo cual tenemos
         $$
         \left|\int_{C_R}f(z)dz\right|\leq \frac{\pi R}{|a|(R-|z_1|)^2}.
         $$
         Como $\lim_{R\to\infty}(\pi R)/(|a|(R-|z_1|)^2)=0$, se sigue que $\lim_{R\to\infty} \int_{C_R}f(z)dz=0$. Con lo anterior
%%%%%%%%%%%%%%%%%%%%%%%%%%%%%%%%%%%%%%%%%%%%%%%%%%%%%%%%%%
         $$
         \lim_{R\to\infty}\int_{-R}^{R}f(x)dx=2\pi i B,
         $$
         es decir, 
         $$
         \lim_{R\to\infty}\left( \int_{-R}^{0}f(x)dx+\int_{0}^{R}f(x)dx\right)=2 \pi i B.
         $$
         Por el criterio de la integral, $\int_{0}^{\infty}f(x)dx$ converge si y solo si $\sum_{n=0}^{\infty}f(n)$ converge, pero vemos que esto es cierto haciendo comparación del límite con la serie convergente $\sum_{n=1}^{\infty}1/(an^2)$. Por tanto, $\lim_{R\to\infty}\int_{0}^{R}f(x)dx$ converge; como $\lim_{R\to\infty}\left( \int_{-R}^{0}f(x)dx+\int_{0}^{R}f(x)dx\right)$ converge, también lo hace $\lim_{R\to\infty}\int_{-R}^{0}f(x)dx$, de modo que
         $$
         \begin{aligned}
            \lim_{R\to\infty}\left( \int_{-R}^{0}f(x)dx+\int_{0}^{R}f(x)dx\right)&=\lim_{R\to\infty}\int_{-R}^{0}f(x)dx+\lim_{R\to\infty}\int_{0}^{\infty}f(x)dx\\
                                                                                               &=\int_{-\infty}^{\infty} f(x)dx,
         \end{aligned}
         $$
         y por tanto
         $$
         \begin{aligned}
            \int_{-\infty}^{\infty}f(x)dx&=2\pi i B\\
                                         &=2\pi i\cdot\sgn(a)\frac{1}{i\sqrt{4ac-b^2}}\\
                                         &=\sgn(a)\frac{2\pi}{\sqrt{4ac-b^2}}.
         \end{aligned}
         $$
      \item[\textbf{II.}] $\int_{-\infty}^{\infty} \frac{dx}{(ax^2+bx+c)^2}$\\
         (Seguimos usando $z_1,z_2,C,\dots$ con el mismo significado del ítem \textbf{I.}) Llamando $g(z)=1/(az^2+bz+c)^2$ tenemos
         $$
         g(z)=\frac{1}{z^2(z-z_1)^2(z-z_2)^2}.
         $$
         Notemos que $r(z):=a^2(z-z_1)^2(z-z_2)^2$ no tiene ceros reales y $z_1$ es su único cero por encima del eje real. Además, como la función $1/(a^2(z-z_2)^2)$ es analítica y no nula en $z_1$, se tiene que $z_1$ es un polo de orden $2$ de $g(z)$, y
         $$
         \begin{aligned}
            D:=&=\mathop{\mathrm{Res}}\limits_{z=z_1}g(z)\\
               &=\frac{d}{dz}\left.\left( \frac{1}{a^2(z-z_2)^2}\right)\right|_{z=z_1}\\
               &=\left. \frac{-2}{a^2(z-z_2)^3}\right|_{z=z_1}\\
               &=\frac{-2}{a^2(z_1-z_2)^3}\\
               &=\frac{-2}{a^2(z_1-\overline{z_1})^3}\\
               &=\frac{-2}{a^2\left( 2i\text{Im}(z_1)\right)^3}\\
               &=\frac{-2}{a^2\left( 2i\frac{\sqrt{4ac-b^2}}{2|a|}\right)^3}\\
               &=\frac{-2}{-\frac{i}{|a|}(4ac-b^2)^{3/2}}\\
               &=-\frac{2i|a|}{(4ac-b^2)^{3/2}}.
         \end{aligned}
         $$
         Como $q(z)$ y $r(z)$ tienen el mismo cero sobre el eje real, podemos integrar $g(z)$ sobre el contorno $C$ para, de forma similar a como se hizo en \textbf{I.} (con la salvedad de que el criterio de comparación del límite se hace con la serie convergente $\sum_{n=1}^{\infty}1/(an^4)$), llegar a
         $$
         \begin{aligned}
            \int_{-\infty}^{\infty}g(x)dx&=2\pi i D\\
                                         &=2\pi i\left( -\frac{2i|a|}{(4ac-b^2)^{3/2}}\right)\\
                                         &=\frac{4\pi|a|}{(4ac-b^2)^{3/2}}.
         \end{aligned}
         $$
   \end{itemize}
  }
\hfill\qedsymbol
