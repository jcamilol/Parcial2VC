\begin{Ejc}
    Suponga que la serie
    $$
    \sum_{n=-\infty}^{\infty} x[n]z^{-n}
    $$
    converge a una función analítica $X(z)$ en algún anillo $R_1<|z|<R_2$. La suma $X(z)$ es llamada la \textbf{z-transformada} de $x[n]$ $(n=0,\pm 1,\pm 2,\dots)$. Use la expresión (5), Sec. 66, para los coeficientes en una serie de Laurent para mostrar que si el anillo contiene la circunferencia unitaria $|z|=1$, entonces la $z$-transformada inversa de $X(z)$ puede escribirse como
    $$
    x[n]=\frac{1}{2\pi}\int_{-\pi}^{\pi} X(e^{i\theta})e^{in\theta}d\theta\hspace{1cm}(n=0,\pm 1, \pm 2,\dots).
    $$
\end{Ejc}
\begin{proof}
    Llamemos $A$ a la región anular $R_1<|z|<R_2$, y $C$ a la circunferencia $|z|=1$ orientada positivamente. Supongamos que $C$ está contenido en $A$. Como $X(z)$ es analítica en $A$ y $C$ es un contorno simple orientado positivamente alrededor de $0$ y contenido en $A$, se sigue que $X(z)$ tiene representación en serie de Laurent
    $$
    X(z)=\sum_{-\infty}^{\infty} c_n z^n, \phantom{1cm}(z\in A)
    $$
    donde, para todo $n\in \mathbb{Z}^{+}$,
    $$
    c_n=\frac{1}{2\pi i}\int_{C}\frac{X(z)}{z^{n+1}} dz.
    $$
    Como además, para todo $z\in A$ tenemos
    $$
    X(z)=\sum_{n=-\infty}^{\infty} x[n]z^{-n},
    $$
    por la unicidad de la representación en serie de Laurent, se sigue que para todo $n\in \mathbb{Z}^{+}$,
    $$
    \begin{aligned}
       x[n]&=c_{-n}\\
           &=\frac{1}{2\pi i}\int_{C}\frac{X(z)}{z^{-n+1}} dz\\
           &=\frac{1}{2\pi i}\int_{C}X(z)z^{n-1}dz.\\
    \end{aligned}
    $$
    Haciendo el cambio de variable $z=e^{i\theta}$, $dz=ie^{i\theta}d\theta$, se tiene
    $$
    \begin{aligned}
       \frac{1}{2\pi i}\int_{C}X(z)z^{n-1}dz&=\frac{1}{2\pi i}\int_{-\pi}^{\pi}X(e^{i\theta})({e^{i\theta}})^{n-1}ie^{i\theta}d\theta\\
                                            &=\frac{1}{2\pi}\int_{-\pi}^{\pi} X(e^{i\theta})e^{in\theta}d\theta,
    \end{aligned}
    $$
    con lo que concluimos
    $$
    x[n]=\frac{1}{2\pi}\int_{-\pi}^{\pi} X(e^{i\theta})e^{in\theta}d\theta\hspace{1cm}(n=0,\pm 1, \pm 2,\dots).
    $$
\end{proof}
