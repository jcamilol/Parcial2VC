\begin{Ejc}
   Calcule $\oint _{|z|=2}\frac{\senh (1/z)}{z-1}dz$.
\end{Ejc}
\begin{proof}
   Asumimos que el contorno $|z|=2$ está orientado positivamente. Llamemos $f(z)=\frac{\senh(1/z)}{z-1}$. Como $\senh$ es una función entera, los únicos puntos en que $f$ no es analítica son $z_0=0$ y $z_1=1$, que se encuentran dentro de $|z|=2$, y por tanto son puntos singulares aislados de $f$. Por el teorema de los residuos de Cauchy tenemos
   $$
   \oint _{|z|=2}\frac{\senh (1/z)}{z-1}dz=2\pi i \sum_{k=0}^{1}\mathop{\mathrm{Res}}\limits_{z=z_k}f(z).
   $$
  \begin{itemize}
     \item Calculemos $\mathop{\mathrm{Res}}\limits_{z=z_0}f(z)$. Sabemos que
        $$
        \senh (z) = z+\frac{z^3}{3!}+\frac{z^5}{5!}+\dots \hspace{1cm}(z\in\mathbb{C}),
        $$
        luego,
        $$
        \senh\left(\frac{1}{z} \right)=\frac{1}{z}+\frac{1}{3!}\frac{1}{z^3}+\frac{1}{5!}\frac{1}{z^5}+\dots \hspace{1cm}(z\in\mathbb{C}-\left\lbrace 0\right\rbrace).
        $$
        También, para $|z|<1$:
        $$
        \begin{aligned}
           \frac{1}{z-1}&=-\frac{1}{1-z}\\
                        &=-\sum_{n=0}^{\infty}z^n\\
                        &=-(1+z+z^2+\dots)\\
                        &=-1-z-z^2-\dots,
        \end{aligned}
        $$
        de modo que para $0<|z|<1$ se tiene
        $$
        \begin{aligned}
           f(z)&=\senh\left( \frac{1}{z}\right)\frac{1}{z-1}\\
               &=\left( \frac{1}{z}+\frac{1}{3!}\frac{1}{z^3}+\frac{1}{5!}\frac{1}{z^5}+\dots\right)\left( -1-z-z^2-\dots\right).
        \end{aligned}
        $$
        En particular, podemos ver que el coeficiente de $1/z$ en la anterior serie es
        $$
        \begin{aligned}
           \mathop{\mathrm{Res}}\limits_{z=z_0}f(z)&=-1-\frac{1}{3!}-\frac{1}{5!}-\frac{1}{7!}-\dots\\
           &=\sum_{n=0}^{\infty}-\frac{1}{(2n-1)!}.\\
        \end{aligned}
        $$
        Sabemos que, para todo $z\in\mathbb{C}$ se tiene $e^z=\sum_{n=0}^{\infty}z^n/n!$, de modo que
        $$
        \begin{aligned}
           e-e^{-1}&=\left( \sum_{n=0}^{\infty}\frac{1}{n!}\right)\left( \sum_{n=0}^{\infty}\frac{(-1)^n}{n!}\right)\\
                   &=\sum_{n=0}^{\infty}\left( 1-(-1)^n\right)\frac{1}{n!}\\
                   &=\frac{2}{1!}+\frac{2}{3!}+\frac{2}{5!}+\dots\\
                   &=\sum_{n=0}^{\infty}\frac{2}{(2n+1)!},
        \end{aligned}
        $$
        y por tanto
        $$
        \frac{e-e^{-1}}{2}=\sum_{n=0}^{\infty}\frac{1}{(2n+1)!}.
        $$
        De lo anterior obtenemos
        $$
        \mathop{\mathrm{Res}}\limits_{z=z_0}f(z)=-\frac{e-e^{-1}}{2}=-\senh (1).
        $$
     \item Calculemos $\mathop{\mathrm{Res}}\limits_{z=z_1}f(z)$. Como $\senh (1/z)$ es analítica y no nula en $z_1=1$, tenemos que $z_1=1$ es un polo simple de $f$ y por tanto $\mathop{\mathrm{Res}}\limits_{z=z_1}f(z)=\senh (1)$.
  \end{itemize}
  Con los ítems anteriores concluimos
  $$
  \begin{aligned}
     \oint _{|z|=2}\frac{\senh (1/z)}{z-1}dz&=2\pi i \sum_{k=0}^{1}\mathop{\mathrm{Res}}\limits_{z=z_k}f(z)\\
                                            &=2\pi i\left( -\senh(1)+\senh(1)\right)\\
                                            &=0.
  \end{aligned}
  $$
\end{proof}
