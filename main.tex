\documentclass[letterpaper]{article} %único necesario para: crear documento, hacer título, espacios, interlineado


\usepackage{vmargin} %necesario para ajustar las márgenes con \setmargins
%\usepackage{euler} %usa la fuente "euler" para las ecuaciones
\usepackage{amsfonts} %para Z estilizada
\usepackage{pifont} %para más estilos de viñetas
\usepackage{amsmath} %necesario para equation*
\usepackage{amsthm} %necesario para proof enviroment
\usepackage{dsfont} %alguna letra caligráfica
\usepackage{tipa} %Epsilon bonito
\usepackage{graphicx} %para manejar imágenes
\usepackage{wrapfig} %para imágenes en modo "wrap"
\usepackage{lipsum} %para usar el texto de relleno
\usepackage{tikz-cd} %para los diagramas conmutativos
\usepackage{amssymb}
\graphicspath{ {./Images/} } %la dirección de la cual se sacan las imágenes
\newtheorem{theorem}{Teorema}
\newtheorem{lemma}{Lema}
\newtheorem{Ejc}{Ejercicio}
\renewcommand*{\proofname}{Prueba} %Muestra "prueba" en lugar de "proof"
\newcommand{\senh}{\normalfont{\text{senh}}}
\newcommand{\sgn}{\normalfont{\text{sgn}}}

%letterpaper: 21.59cm x 27.94cm
\setmargins{1.8 cm} %margen izquierdo
{1.5 cm} %margen Superior
{17.5 cm} %anchura del texto
{23cm} %altura del texto
{0pt} %altura de los encabezados
{1.5cm} %espacio entre el texto y los encabezados
{0cm} %altura del pie de página
{2 cm} %espacio entre el texto y el pie de página


%\pagenumbering{gobble} %Quita la numeración de las páginas
\renewcommand{\baselinestretch}{1.5} %Aumenta el interlineado a n veces el automático
\relpenalty=9999 %Evita que se rompan las ecuaciones en el cambio de renglón
\binoppenalty=9999

\title{
  \vspace{-1.5cm}
  \textsc{
    \Large{Variable Compleja\\ Segundo Examen Parcial}\\ \vspace{0.2cm}
    \large{Julio 30 de 2024\\ \vspace{1.5cm} {Alexis} \\ \vspace{0.2cm} {Juan Camilo Lozano Suárez}}\\
  }
}

\date{}

\begin{document}

    \maketitle
    \begin{exercise}
    Suponga que la serie
    $$
    \sum_{n=-\infty}^{\infty} x[n]z^{-n}
    $$
    converge a una función analítica $X(z)$ en algún anillo $R_1<|z|<R_2$. La suma $X(z)$ es llamada la \textbf{z-transformada} de $x[n]$ ($n=0,\pm 1,\pm 2,\dots$). Use la expresión (5), Sec. 66, para los coeficientes en una serie de Laurent para mostrar que si el anillo contiene la circunferencia unitaria $|z|=1$, entonces la $z$-transformada inversa de $X(z)$ puede escribirse como
    $$
    x[n]=\frac{1}{2\pi}\int_{-\pi}^{\pi} X(e^{i\theta})e^{in\theta}d\theta\hspace{1cm}(n=0,\pm 1, \pm 2,\dots).
    $$
\end{exercise}
\begin{proof}
    Llamemos $A$ a la región anular $R_1<|z|<R_2$, y $C$ a la circunferencia $|z|=1$ orientada positivamente. Supongamos que $C$ está contenido en $A$. Como $X(z)$ es analítica en $A$ y $C$ es un contorno simple orientado positivamente alrededor de $0$ y contenido en $A$, se sigue que $X(z)$ tiene representación en serie de Laurent
    $$
    X(z)=\sum_{-\infty}^{\infty} c_n z^n, \phantom{1cm}(z\in A)
    $$
    donde, para todo $n\in \mathbb{Z}^{+}$,
    $$
    c_n=\frac{1}{2\pi i}\int_{C}\frac{X(z)}{z^{n+1}} dz.
    $$
    Como además, para todo $z\in A$ tenemos
    $$
    X(z)=\sum_{n=-\infty}^{\infty} x[n]z^{-n},
    $$
    por la unicidad de la representación en serie de Laurent, se sigue que para todo $n\in \mathbb{Z}^{+}$,
    $$
    \begin{aligned}
       x[n]&=c_{-n}\\
           &=\frac{1}{2\pi i}\int_{C}\frac{X(z)}{z^{-n+1}} dz\\
           &=\frac{1}{2\pi i}\int_{C}X(z)z^{n-1}dz.\\
    \end{aligned}
    $$
    Haciendo el cambio de variable $z=e^{i\theta}$, $dz=ie^{i\theta}d\theta$, se tiene
    $$
    \begin{aligned}
       \frac{1}{2\pi i}\int_{C}X(z)z^{n-1}dz&=\frac{1}{2\pi i}\int_{-\pi}^{\pi}X(e^{i\theta})({e^{i\theta}})^{n-1}ie^{i\theta}d\theta\\
                                            &=\frac{1}{2\pi}\int_{-\pi}^{\pi} X(e^{i\theta})e^{in\theta}d\theta,
    \end{aligned}
    $$
    con lo que concluimos
    $$
    x[n]=\frac{1}{2\pi}\int_{-\pi}^{\pi} X(e^{i\theta})e^{in\theta}d\theta\hspace{1cm}(n=0,\pm 1, \pm 2,\dots).
    $$
\end{proof}

    \begin{Ejc}
   (Segundo punto)
\end{Ejc}
\begin{proof}
   (Prueba segundo punto) 
\end{proof}

    \begin{Ejc}
   Calcule $\oint _{|z|=2}\frac{\senh (1/z)}{z-1}dz$.
\end{Ejc}
\textit{Solución.} \normalfont{Asumimos que el contorno $|z|=2$ está orientado positivamente. Llamemos $f(z)=\frac{\senh(1/z)}{z-1}$. Como $\senh$ es una función entera, los únicos puntos en que $f$ no es analítica son $z_0=0$ y $z_1=1$, que se encuentran dentro de $|z|=2$, y por tanto son puntos singulares aislados de $f$. Por el teorema de los residuos de Cauchy tenemos
   $$
   \oint _{|z|=2}\frac{\senh (1/z)}{z-1}dz=2\pi i \sum_{k=0}^{1}\mathop{\mathrm{Res}}\limits_{z=z_k}f(z).
   $$
  \begin{itemize}
     \item Calculemos $\mathop{\mathrm{Res}}\limits_{z=z_0}f(z)$. Sabemos que
        $$
        \senh (z) = z+\frac{z^3}{3!}+\frac{z^5}{5!}+\dots \hspace{1cm}(z\in\mathbb{C}),
        $$
        luego,
        $$
        \senh\left(\frac{1}{z} \right)=\frac{1}{z}+\frac{1}{3!}\frac{1}{z^3}+\frac{1}{5!}\frac{1}{z^5}+\dots \hspace{1cm}(z\in\mathbb{C}-\left\lbrace 0\right\rbrace).
        $$
        También, para $|z|<1$:
        $$
        \begin{aligned}
           \frac{1}{z-1}&=-\frac{1}{1-z}\\
                        &=-\sum_{n=0}^{\infty}z^n\\
                        &=-(1+z+z^2+\dots)\\
                        &=-1-z-z^2-\dots,
        \end{aligned}
        $$
        de modo que para $0<|z|<1$ se tiene
        $$
        \begin{aligned}
           f(z)&=\senh\left( \frac{1}{z}\right)\frac{1}{z-1}\\
               &=\left( \frac{1}{z}+\frac{1}{3!}\frac{1}{z^3}+\frac{1}{5!}\frac{1}{z^5}+\dots\right)\left( -1-z-z^2-\dots\right).
        \end{aligned}
        $$
        En particular, podemos ver que el coeficiente de $1/z$ en la anterior serie es
        $$
        \begin{aligned}
           \mathop{\mathrm{Res}}\limits_{z=z_0}f(z)&=-1-\frac{1}{3!}-\frac{1}{5!}-\frac{1}{7!}-\dots\\
           &=\sum_{n=0}^{\infty}-\frac{1}{(2n-1)!}.\\
        \end{aligned}
        $$
        Sabemos que, para todo $z\in\mathbb{C}$ se tiene $e^z=\sum_{n=0}^{\infty}z^n/n!$, de modo que
        $$
        \begin{aligned}
           e-e^{-1}&=\left( \sum_{n=0}^{\infty}\frac{1}{n!}\right)\left( \sum_{n=0}^{\infty}\frac{(-1)^n}{n!}\right)\\
                   &=\sum_{n=0}^{\infty}\left( 1-(-1)^n\right)\frac{1}{n!}\\
                   &=\frac{2}{1!}+\frac{2}{3!}+\frac{2}{5!}+\dots\\
                   &=\sum_{n=0}^{\infty}\frac{2}{(2n+1)!},
        \end{aligned}
        $$
        y por tanto
        $$
        \frac{e-e^{-1}}{2}=\sum_{n=0}^{\infty}\frac{1}{(2n+1)!}.
        $$
        De lo anterior obtenemos
        $$
        \mathop{\mathrm{Res}}\limits_{z=z_0}f(z)=-\frac{e-e^{-1}}{2}=-\senh (1).
        $$
     \item Calculemos $\mathop{\mathrm{Res}}\limits_{z=z_1}f(z)$. Como $\senh (1/z)$ es analítica y no nula en $z_1=1$, tenemos que $z_1=1$ es un polo simple de $f$ y por tanto $\mathop{\mathrm{Res}}\limits_{z=z_1}f(z)=\senh (1)$.
  \end{itemize}
  Con los ítems anteriores concluimos
  $$
  \begin{aligned}
     \oint _{|z|=2}\frac{\senh (1/z)}{z-1}dz&=2\pi i \sum_{k=0}^{1}\mathop{\mathrm{Res}}\limits_{z=z_k}f(z)\\
                                            &=2\pi i\left( -\senh(1)+\senh(1)\right)\\
                                            &=0.
  \end{aligned}
  $$
}
\hfill\qedsymbol

    \begin{Ejc}
\end{Ejc}
\textit{Solución.}{
  }
\hfill\qedsymbol

    \begin{Ejc}
   Calcule usando residuos:
   $$
   \int_{-\infty}^{\infty}\frac{dx}{ax^2+bx+c} 
   $$
   y
   $$
   \int_{-\infty}^{\infty}\frac{dx}{\left( ax^2+bx+c\right)^2}
   $$
   donde $a, b$ y $c$ son reales tales que $b^2<4ac$.
\end{Ejc}
\textit{Solución.}{
   \begin{itemize}
      \item[\textbf{I.}]$\int_{-\infty}^{\infty}\frac{dx}{ax^2+bx+c}$\\
         Llamemos $f(z)=1/(az^2+bz+c)$. Como $4ac>b^2\geq0$, tenemos $a\neq 0$. Así, tomando
         $$
         z_1=-\frac{b}{2a}+i\frac{\sqrt{4ac-b^2}}{2|a|}\phantom{000}\text{y}\phantom{000}z_2=-\frac{b}{2|a|}-i\frac{\sqrt{4ac-b^2}}{2a}=\overline{z_1},
         $$
         tenemos
         $$
         f(z)=\frac{1}{a(z-z_1)(z-z_2)}.
         $$
         Llamando $q(z)=a(z-z_1)(z-z_2)$, como $4ac-b^2\neq 0$, tenemos que $q(z)$ no tiene ceros reales, y $z_1$ es su único cero sobre el eje real. Además, como la función $1/(a(z-z_2))$ es analítica y no nula en $z_1$, se tiene que $z_1$ es un polo simple de $f(z)$, y
         $$
         \begin{aligned}
            B:=&=\mathop{\mathrm{Res}}\limits_{z=z_1}f(z)\\
               &=\frac{1}{a(z_1-z_2)}\\
               &=\frac{1}{a(z_1-\overline{z_1})}\\
               &=\frac{1}{2ia\cdot\text{Im}(z_1)}\\
               &=\frac{1}{2ia\frac{\sqrt{4ac-b^2}}{2|a|}}\\
               &=\frac{|a|}{ia\sqrt{4ac-b^2}}\\
               &=\sgn(a)\frac{1}{i\sqrt{4ac-b^2}}.
         \end{aligned}
         $$
         Tomamos $R>|z_1|$ y llamamos $C_R$ a la curva $Re^{i\theta}$ con $\theta\in[0,\pi]$, y $C$ a la curva que consta del eje real de $-R$ a $R$ junto a $C_R$. Por el teorema de los residuos de Cauchy tenemos
         $$
         \int_{-R}^{R}f(x)dx+\int_{C_R}f(z)dz=\int_{C}f(z)dz=2\pi i B,
         $$
         y por tanto,
         $$
         \lim_{R\to\infty}\int_{-R}^{R}f(x)dx=2\pi i B + \lim_{R\to\infty}\int_{C_R}f(z)dz.
         $$
         Por el lema de Jordan, $\lim_{R\to\infty}\int_{C_R} f(z)dz=0$, así que
         $$
         \lim_{R\to\infty}\int_{-R}^{R}f(x)dx=2\pi i B,
         $$
         es decir, 
         $$
         \lim_{R\to\infty}\left( \int_{-R}^{0}f(x)dx+\int_{0}^{R}f(x)dx\right)=2 \pi i B.
         $$
         Por el criterio de la integral, $\int_{0}^{\infty}f(x)dx$ converge si y solo si $\sum_{n=0}^{\infty}f(n)$ converge, pero vemos que esto es cierto haciendo comparación del límite con la serie convergente $\sum_{n=1}^{\infty}1/(an^2)$. Por tanto, $\lim_{R\to\infty}\int_{0}^{R}f(x)dx$ converge; como $\lim_{R\to\infty}\left( \int_{-R}^{0}f(x)dx+\int_{0}^{R}f(x)dx\right)$ converge, también lo hace $\lim_{R\to\infty}\int_{-R}^{0}f(x)dx$, de modo que
         $$
         \begin{aligned}
            \lim_{R\to\infty}\left( \int_{-R}^{0}f(x)dx+\int_{0}^{R}f(x)dx\right)&=\lim_{R\to\infty}\int_{-R}^{0}f(x)dx+\lim_{R\to\infty}\int_{0}^{\infty}f(x)dx\\
                                                                                               &=\int_{-\infty}^{\infty} f(x)dx,
         \end{aligned}
         $$
         y por tanto
         $$
         \begin{aligned}
            \int_{-\infty}^{\infty}f(x)dx&=2\pi i B\\
                                         &=2\pi i\cdot\sgn(a)\frac{1}{i\sqrt{4ac-b^2}}\\
                                         &=\sgn(a)\frac{2\pi}{\sqrt{4ac-b^2}}.
         \end{aligned}
         $$
      \item[\textbf{II.}] $\int_{-\infty}^{\infty} \frac{dx}{(ax^2+bx+c)^2}$\\
         (Seguimos usando $z_1,z_2,C,\dots$ con el mismo significado del ítem \textbf{I.}) Llamando $g(z)=1/(az^2+bz+c)^2$ tenemos
         $$
         g(z)=\frac{1}{z^2(z-z_1)^2(z-z_2)^2}.
         $$
         Notemos que $r(z):=a^2(z-z_1)^2(z-z_2)^2$ no tiene ceros reales y $z_1$ es su único cero por encima del eje real. Además, como la función $1/(a^2(z-z_2)^2)$ es analítica y no nula en $z_1$, se tiene que $z_1$ es un polo de orden $2$ de $g(z)$, y
         $$
         \begin{aligned}
            D:=&=\mathop{\mathrm{Res}}\limits_{z=z_1}g(z)\\
               &=\frac{d}{dz}\left.\left( \frac{1}{a^2(z-z_2)^2}\right)\right|_{z=z_1}\\
               &=\left. \frac{-2}{a^2(z-z_2)^3}\right|_{z=z_1}\\
               &=\frac{-2}{a^2(z_1-z_2)^3}\\
               &=\frac{-2}{a^2(z_1-\overline{z_1})^3}\\
               &=\frac{-2}{a^2\left( 2i\text{Im}(z_1)\right)^3}\\
               &=\frac{-2}{a^2\left( 2i\frac{\sqrt{4ac-b^2}}{2|a|}\right)^3}\\
               &=\frac{-2}{-\frac{i}{|a|}(4ac-b^2)^{3/2}}\\
               &=-\frac{2i|a|}{(4ac-b^2)^{3/2}}.
         \end{aligned}
         $$
         Como $q(z)$ y $r(z)$ tienen el mismo cero sobre el eje real, podemos integrar $g(z)$ sobre el contorno $C$ para, de forma similar a como se hizo en \textbf{I.} (con la salvedad de que el criterio de comparación del límite se hace con la serie convergente $\sum_{n=1}^{\infty}1/(an^4)$), llegar a
         $$
         \begin{aligned}
            \int_{-\infty}^{\infty}g(x)dx&=2\pi i D\\
                                         &=2\pi i\left( -\frac{2i|a|}{(4ac-b^2)^{3/2}}\right)\\
                                         &=\frac{4\pi|a|}{(4ac-b^2)^{3/2}}.
         \end{aligned}
         $$
   \end{itemize}
  }
\hfill\qedsymbol

\end{document}
